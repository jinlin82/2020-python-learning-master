\documentclass[ignorenonframetext,11pt,xcolor=dvipsnames,hyperref={colorlinks,allcolors=.,urlcolor=blue, citecolor=violet, bookmarksdepth=4},aspectratio=1610]{beamer}
\setbeamertemplate{caption}[numbered]
\setbeamertemplate{caption label separator}{: }
\setbeamercolor{caption name}{fg=normal text.fg}
\beamertemplatenavigationsymbolsempty
\newcommand\hmmax{0} %% 防止Too many math alphabets used in version normal.
\newcommand\bmmax{0} %% 防止Too many math alphabets used in version normal.
\usepackage{lmodern,bm}   % 必需出现在amsmath等包前面,否则会出错
\usepackage{amssymb,amsmath}
\usepackage{ifxetex,ifluatex}
\usepackage{fixltx2e} % provides \textsubscript
\ifnum 0\ifxetex 1\fi\ifluatex 1\fi=0 % if pdftex
  \usepackage[T1]{fontenc}
  \usepackage[utf8]{inputenc}
\else % if luatex or xelatex
  \ifxetex
    \usepackage{mathspec}
  \else
    \usepackage{fontspec}
  \fi
  \defaultfontfeatures{Ligatures=TeX,Scale=MatchLowercase}
\fi
% use upquote if available, for straight quotes in verbatim environments
\IfFileExists{upquote.sty}{\usepackage{upquote}}{}
% use microtype if available
\IfFileExists{microtype.sty}{%
\usepackage{microtype}
\UseMicrotypeSet[protrusion]{basicmath} % disable protrusion for tt fonts
}{}

%\newif\ifbibliography

%%\usepackage{natbib}
%\bibliographystyle{plainnat}
%
%
\hypersetup{
            pdftitle={Numpy 用法},
            pdfauthor={Jin},
            pdfborder={0 0 0},
            breaklinks=true}
\urlstyle{same}  % don't use monospace font for urls
\usepackage{longtable,booktabs}
\usepackage{caption}
% These lines are needed to make table captions work with longtable:
\makeatletter
\def\fnum@table{\tablename~\thetable}
\makeatother

% Prevent slide breaks in the middle of a paragraph:
\widowpenalties 1 10000
\raggedbottom

\AtBeginPart{
  \let\insertpartnumber\relax
  \let\partname\relax
  \frame{\partpage}
}
\AtBeginSection{
  \ifbibliography
  \else
    \let\insertsectionnumber\relax
    \let\sectionname\relax
    \frame{\sectionpage}
  \fi
}
\AtBeginSubsection{
  \let\insertsubsectionnumber\relax
  \let\subsectionname\relax
  \frame{\subsectionpage}
}

\setlength{\parindent}{0pt}
\setlength{\parskip}{6pt plus 2pt minus 1pt}
\setlength{\emergencystretch}{3em}  % prevent overfull lines
\providecommand{\tightlist}{%
  \setlength{\itemsep}{0pt}\setlength{\parskip}{0pt}}
\setcounter{secnumdepth}{0}

\usepackage[BoldFont,SlantFont]{xeCJK}

\setCJKmainfont[BoldFont=Microsoft YaHei]{SimSun}
\setCJKsansfont[BoldFont=Microsoft YaHei Bold]{Microsoft YaHei}
\setCJKromanfont{SimSun}
\setCJKmonofont{SimSun}

\IfFileExists{C:/Windows/Fonts/AdobeSongStd-Light.otf}{
  \setCJKfamilyfont{song}{AdobeSongStd-Light}}{
  \setCJKfamilyfont{song}{SimSun.ttc}}

\IfFileExists{C:/Windows/Fonts/AdobeHeitiStd-Regular.otf}{
  \setCJKfamilyfont{hei}{AdobeHeitiStd-Regular}}{
  \setCJKfamilyfont{hei}{SimHei.ttf}}

\IfFileExists{C:/Windows/Fonts/AdobeKaitiStd-Regular.otf}{
  \setCJKfamilyfont{kai}{AdobeKaitiStd-Regular}}{
  \setCJKfamilyfont{kai}{SimKai.ttf}}

\IfFileExists{C:/Windows/Fonts/AdobeFangsongStd-Regular.otf}{
  \setCJKfamilyfont{fs}{AdobeFangsongStd-Regular}}{
  \setCJKfamilyfont{fs}{SimFang.ttf}}

\author[Jin]{\CJKfamily{kai}Jin \\ jinlin@zuel.edu.cn \\}
\institute[中南财经政法大学统计与数学学院]{\normalsize\CJKfamily{kai}中南财经政法大学统计与数学学院}

\date{\today}
\date{2020年6月}





\renewcommand{\contentsname}{\centerline{\textcolor{violet}{目 \ \ 录}}}    % 将Contents改为目录
\renewcommand{\abstractname}{摘 \ \ 要}      % 将Abstract改为摘要
\renewcommand{\refname}{参考文献}            % 将Reference改为参考文献
\renewcommand\tablename{表}
\renewcommand\figurename{图}
\renewcommand{\today}{\number\year 年 \number\month 月 \number\day 日}

\PassOptionsToPackage{dvipsnames}{xcolor}
\PassOptionsToPackage{colorlinks=true,citecolor=blue, urlcolor=blue, linkcolor=violet, bookmarksdepth=4,bookmarksopen=true,bookmarksopenlevel=2}{hyperref}

\usepackage{lscape}
\usepackage{indentfirst}
\usepackage{textcomp}                      % provide many text symbols
\usepackage{setspace}                      % 各种间距设置


% ---------------------------------Table------------------------------
\usepackage{booktabs}
\usepackage{array}                         % 提供表格中每一列的宽度及位置支持
\usepackage{multirow}
\usepackage{rotating}
\newcolumntype{L}[1]{>{\raggedright\let\newline\\\arraybackslash\hspace{0pt}}m{#1}}
\newcolumntype{C}[1]{>{\centering\let\newline\\\arraybackslash\hspace{0pt}}m{#1}}
\newcolumntype{R}[1]{>{\raggedleft\let\newline\\\arraybackslash\hspace{0pt}}m{#1}}

%\sloppy
%\linespread{1.0}                           % 设置行距
\setlength{\parindent}{22pt}
%\setlength{\parskip}{1ex plus 0.5ex minus 0.2ex}


%% 参考文献
\usepackage{gbt7714}
\usepackage{natbib}
\setlength{\bibsep}{2pt}


\usepackage[utf8]{inputenc}
% Package fontenc omitted
% Package fixltx2e omitted
\usepackage{graphicx}
% Package longtable omitted
% Package float omitted
% Package wrapfig omitted
\usepackage{soul}
% Package textcomp omitted
\usepackage{marvosym}
\usepackage{wasysym}
\usepackage{latexsym}
\usepackage{amssymb}
% Package hyperref omitted
\usepackage{listings}
\newcommand{\passthrough}[1]{#1}
\usepackage{tikz}

						   
\setmonofont{Consolas} % listings 中支持 consolas 字体,必需配合上面usepackage{fontenc} 中不出现[T1]才可以

\lstset{numbers=left, numberstyle=\ttfamily\tiny\color{Gray}, stepnumber=1, numbersep=8pt,
  frame=leftline,
  framexleftmargin=0mm,
  rulecolor=\color{CadetBlue},
  backgroundcolor=\color{Periwinkle!20},
  stringstyle=\color{CadetBlue},
  flexiblecolumns=false,
  aboveskip=5pt,
  belowskip=0pt,
  language=R,
  basicstyle=\ttfamily\footnotesize,
  columns=flexible,
  keepspaces=true,
  breaklines=true,
  extendedchars=true,
  texcl=false,  % 必须设置为false设置为true的时候 R 代码中不能含有多个注释符号 #
  upquote=true, % 设置 引号为竖引号,但必需配合 上面 fontenc T1 使用,fontenc T1 又不能使用 consolas,所以冲突
  showstringspaces=false,
  keywordstyle=\bfseries,
  keywordstyle=\color{Purple},
  xleftmargin=20pt,
  xrightmargin=10pt,
  morecomment=[s]{\#}{\#},
  commentstyle=\color{OliveGreen!60}\scriptsize,
  tabsize=4}

\tolerance=1000
\usetheme{default}
\setcounter{secnumdepth}{4}

\usetheme{default}
\useinnertheme[shadow]{rounded}
\useoutertheme{infolines}
\usecolortheme{seahorse}
\setbeamercolor{frametitle}{fg=Blue, bg=white}
\setbeamercolor{titlelike}{parent=structure}
\setbeamertemplate{caption}[numbered]
\setbeamertemplate{section in toc shaded}[default][50]
\setbeamertemplate{frametitle continuation}[from second][(续)] % 改变
\setbeamertemplate{subsection in toc shaded}[default][20]
\setbeamertemplate{subsection in toc}[square]
\logo{\includegraphics[height=0.6cm,width=0.6cm]{znufelogo.jpg}}
\setbeamercovered{transparent}
\setCJKmainfont[BoldFont={* Bold}]{Microsoft YaHei}
\usefonttheme[onlylarge]{structuresmallcapsserif}
\usefonttheme[onlymath]{serif}
\setbeamertemplate{frametitle}{\bfseries\insertframetitle\par\vskip-6pt}
\setbeamerfont{block title}{shape=\normalfont, series=\bfseries}
\setbeamercolor{block title}{fg=violet}
\AtBeginSection[]
{
\setcounter{tocdepth}{2}
\frame[shrink=5]{\tableofcontents[currentsection, hideothersubsections]}
}
\AtBeginSubsection[] % Do nothing for \subsection*
{
\begin{frame}<beamer>
\frametitle{}
\large \tableofcontents[currentsection, sectionstyle=shaded/hide, subsectionstyle=show/shaded/hide]
\end{frame}
}
\setlength{\parskip}{1ex plus 0.5ex minus 0.2ex}
\everydisplay{\color{blue}}


\title{Numpy 用法}

\makeatletter
\@ifpackageloaded{subfig}{}{\usepackage{subfig}}
\@ifpackageloaded{caption}{}{\usepackage{caption}}
\captionsetup[subfloat]{margin=0.5em}
\AtBeginDocument{%
\renewcommand*\figurename{Figure}
\renewcommand*\tablename{Table}
}
\AtBeginDocument{%
\renewcommand*\listfigurename{List of Figures}
\renewcommand*\listtablename{List of Tables}
}
\@ifpackageloaded{float}{}{\usepackage{float}}
\floatstyle{ruled}
\@ifundefined{c@chapter}{\newfloat{codelisting}{h}{lop}}{\newfloat{codelisting}{h}{lop}[chapter]}
\floatname{codelisting}{Listing}
\newcommand*\listoflistings{\listof{codelisting}{List of Listings}}
\makeatother

\begin{document}
\frame{\titlepage}

\begin{frame}{\textcolor{violet}{\normalfont\Large \CJKfamily{kai}大\ \ 纲 }} \textcolor{violet}{}
\tableofcontents[hideallsubsections]
\end{frame}

\hypertarget{introduction}{%
\section{Introduction}\label{introduction}}

\begin{frame}{Facts}
\protect\hypertarget{facts}{}

\begin{enumerate}
\tightlist
\item
  Initial release: As Numeric, 1995; as NumPy, 2006
\item
  Stable release: 1.11.2 / 3 October 2016;
\item
  Website: \url{http://www.numpy.org}
\item
  History: \url{https://en.wikipedia.org/wiki/NumPy}
\end{enumerate}

\end{frame}

\begin{frame}{What is NumPy?}
\protect\hypertarget{what-is-numpy}{}

\begin{enumerate}
\item
  NumPy is the fundamental package for scientific computing in Python.
\item
  It is a Python library that provides a multidimensional array object,
  various derived objects (such as masked arrays and matrices), and an
  assortment of routines for fast operations on arrays, including
\item
  mathematical,
\item
  logical,
\item
  shape manipulation,
\item
  sorting, selecting, I/O,
\item
  discrete Fourier transforms, basic linear algebra,
\item
  basic statistical operations, random simulation and much more.
\end{enumerate}

\end{frame}

\begin{frame}{NumPy}
\protect\hypertarget{numpy}{}

\begin{enumerate}
\tightlist
\item
  At the core of the NumPy package, is the \textbf{ndarray} object.
\item
  This encapsulates n-dimensional arrays of homogeneous data types, with
  many operations being performed in compiled code for performance.
\item
  The points about sequence size and speed are particularly important in
  scientific computing.
\end{enumerate}

\end{frame}

\begin{frame}{ndarray object}
\protect\hypertarget{ndarray-object}{}

\begin{enumerate}
\tightlist
\item
  Vectorization:Vectorization describes the absence of any explicit
  looping, indexing, etc., in the code - these things are taking place,
  of course, just ``behind the scenes'' in optimized, pre-compiled C
  code.
\item
  Broadcasting:Broadcasting is the term used to describe the implicit
  element-by-element behavior of operations.
\item
  NumPy fully supports an object-oriented approach with ndarray. ndarray
  is a class, possessing numerous methods and attributes.
\end{enumerate}

\end{frame}

\hypertarget{basics}{%
\section{Basics}\label{basics}}

\hypertarget{array-creation}{%
\subsection{Array Creation}\label{array-creation}}

\begin{frame}[fragile]{\passthrough{\lstinline!array!} function}
\protect\hypertarget{array-function}{}

\begin{enumerate}
\tightlist
\item
  create an array from a regular Python list or tuple using the
  \textbf{\passthrough{\lstinline!array!}} function. The type of the
  resulting array is deduced from the type of the elements in the
  sequences.
\item
  array transforms sequences of sequences into two-dimensional arrays,
  sequences of sequences of sequences into three-dimensional arrays, and
  so on.
\item
  Often, the elements of an array are originally unknown, but its size
  is known.
\end{enumerate}

\end{frame}

\begin{frame}[fragile]{arrays with initial placeholder content}
\protect\hypertarget{arrays-with-initial-placeholder-content}{}

\begin{enumerate}
\tightlist
\item
  The function \textbf{\passthrough{\lstinline!zeros!}} creates an array
  full of zeros,
\item
  the function \textbf{\passthrough{\lstinline!ones!}} creates an array
  full of ones,
\item
  the function \textbf{\passthrough{\lstinline!empty!}} creates an array
  whose initial content is random and depends on the state of the
  memory.
\item
  the function \passthrough{\lstinline!diag!} creates the diagonal
  array,
\item
  the function \textbf{\passthrough{\lstinline!eye!}} or
  \textbf{\passthrough{\lstinline!identity!}} creates an array with ones
  on the diagonal and zeros elsewhere.
\end{enumerate}

\end{frame}

\begin{frame}[fragile]{\passthrough{\lstinline!arange!} and
\passthrough{\lstinline!linspace!} function}
\protect\hypertarget{arange-and-linspace-function}{}

\begin{enumerate}
\tightlist
\item
  \textbf{\passthrough{\lstinline!arange!}}:
\item
  \textbf{\passthrough{\lstinline!linspace!}}:
\end{enumerate}

\end{frame}

\begin{frame}{attributes of an ndarray object}
\protect\hypertarget{attributes-of-an-ndarray-object}{}

\begin{enumerate}
\tightlist
\item
  ndarray.ndim
\item
  ndarray.shape
\item
  ndarray.size
\item
  ndarray.dtype
\item
  ndarray.itemsize
\item
  ndarray.data
\end{enumerate}

\end{frame}

\hypertarget{basic-operations}{%
\subsection{Basic Operations}\label{basic-operations}}

\begin{frame}{Basic Operations}
\protect\hypertarget{basic-operations-1}{}

\begin{enumerate}
\tightlist
\item
  Arithmetic operators on arrays apply elementwise. A new array is
  created and filled with the result.
\item
  The matrix product can be performed using the dot function or method:
\item
  Many unary operations, such as computing the sum of all the elements
  in the array, are implemented as methods of the ndarray class.
\item
  by specifying the axis parameter you can apply an operation along the
  specified axis of an array(类似于 R 中的 apply 函数)
\end{enumerate}

\end{frame}

\begin{frame}[fragile]{例子}
\protect\hypertarget{section}{}

\begin{lstlisting}[language=Python]
import numpy as np
a=np.arange(4)
b=np.array([2,5,8,9])
a*b
\end{lstlisting}

\begin{lstlisting}[language=Python]
A=np.arange(12).reshape(3,4)
B=np.arange(13,25).reshape(4,3)
np.dot(A, B)
\end{lstlisting}

\begin{lstlisting}[language=Python]
A.dot(B)
\end{lstlisting}

\begin{lstlisting}[language=Python]
A.sum()
\end{lstlisting}

\begin{lstlisting}[language=Python]
A.sum(axis=0)
\end{lstlisting}

\begin{lstlisting}[language=Python]
A.sum(axis=1)
\end{lstlisting}

\begin{lstlisting}[language=Python]
"""END"""
\end{lstlisting}

\end{frame}

\begin{frame}[fragile]{Universal Functions}
\protect\hypertarget{universal-functions}{}

\begin{enumerate}
\tightlist
\item
  NumPy provides familiar mathematical functions such as sin, cos, and
  exp.
\item
  In NumPy, these are called "universal functions"(ufunc).
\item
  Within NumPy, these functions operate elementwise on an array,
  producing an array as output.
\end{enumerate}

\begin{lstlisting}[language=Python]
A=np.arange(12).reshape(3,4)
np.exp(A)
\end{lstlisting}

\begin{lstlisting}[language=Python]
np.sqrt(A)
\end{lstlisting}

\begin{lstlisting}[language=Python]
"""END"""
\end{lstlisting}

\end{frame}

\hypertarget{indexing-slicing-and-iterating}{%
\subsection{Indexing, Slicing and
Iterating}\label{indexing-slicing-and-iterating}}

\begin{frame}[fragile]{Indexing, Slicing and Iterating}
\protect\hypertarget{indexing-slicing-and-iterating-1}{}

\begin{enumerate}
\tightlist
\item
  One-dimensional arrays can be indexed, sliced and iterated over, much
  like lists and other Python sequences.
\end{enumerate}

\begin{lstlisting}[language=Python]
x=np.arange(12)**2
x[3]
\end{lstlisting}

\begin{lstlisting}[language=Python]
x[2:6]
\end{lstlisting}

\begin{lstlisting}[language=Python]
x[7:]
\end{lstlisting}

\begin{lstlisting}[language=Python]
x[::-1]
\end{lstlisting}

\begin{lstlisting}[language=Python]
x[9:2:-3]
\end{lstlisting}

\begin{lstlisting}[language=Python]
"""END"""
\end{lstlisting}

\end{frame}

\begin{frame}[fragile]{Indexing, Slicing and Iterating}
\protect\hypertarget{indexing-slicing-and-iterating-2}{}

\begin{enumerate}
\tightlist
\item
  Multidimensional arrays can have one index per axis. These indices are
  given in a tuple separated by commas.
\item
  When fewer indices are provided than the number of axes, the missing
  indices are considered complete slices.
\end{enumerate}

\begin{lstlisting}[language=Python]
A=np.arange(24).reshape(4,6)
A[2,3]
\end{lstlisting}

\begin{lstlisting}[language=Python]
A[1:3, 2:4]
\end{lstlisting}

\begin{lstlisting}[language=Python]
A[1]
\end{lstlisting}

\begin{lstlisting}[language=Python]
A[:, 2:4]
\end{lstlisting}

\begin{lstlisting}[language=Python]
A[..., 3]
\end{lstlisting}

\begin{lstlisting}[language=Python]
"""END"""
\end{lstlisting}

\end{frame}

\begin{frame}[fragile]{Indexing, Slicing and Iterating}
\protect\hypertarget{indexing-slicing-and-iterating-3}{}

\begin{enumerate}
\tightlist
\item
  The dots (\ldots) represent as many colons as needed to produce a
  complete indexing tuple.
\item
  For example, if x is an array with 5 axes, then
\end{enumerate}

\begin{itemize}
\tightlist
\item
  \passthrough{\lstinline!x[1, 2, ...]!} is equivalent to
  \passthrough{\lstinline!x[1, 2, :, :, :]!},
\item
  \passthrough{\lstinline!x[..., 3]!} to
  \passthrough{\lstinline!x[:, : ,: ,:, 3]!}
\item
  \passthrough{\lstinline!x[4, ..., 5, :]!} to
  \passthrough{\lstinline!x[4, :, :, 5, :]!}
\end{itemize}

\end{frame}

\begin{frame}[fragile]{Indexing, Slicing and Iterating}
\protect\hypertarget{indexing-slicing-and-iterating-4}{}

\begin{enumerate}
\tightlist
\item
  Iterating over multidimensional arrays is done with respect to the
  first axis
\item
  if one wants to perform an operation on each element in the array, one
  can use the flat attribute which is an iterator over all the elements
  of the array
\end{enumerate}

\begin{lstlisting}[language=Python]
import numpy as np
A=np.arange(24).reshape(4,6)
for i in A:
    """打印A的各行"""
    print(i) 
\end{lstlisting}

\begin{lstlisting}[language=Python]
for i in A.flat:
    """打印A中的每个元素"""
    print(i)
\end{lstlisting}

\begin{lstlisting}[language=Python]
"""END"""
\end{lstlisting}

\end{frame}

\hypertarget{shape-manipulation}{%
\section{Shape Manipulation}\label{shape-manipulation}}

\begin{frame}[fragile]{Changing the shape of an array}
\protect\hypertarget{changing-the-shape-of-an-array}{}

\begin{enumerate}
\tightlist
\item
  An array has a shape given by the number of elements along each axis
\item
  The shape of an array can be changed with various commands.
\item
  Note that the following three commands all return a modified array,
  but do not change the original array:
\end{enumerate}

\begin{enumerate}
\tightlist
\item
  \passthrough{\lstinline!ndarray.ravel(), ndarray.T, ndarry.reshape!}

  \begin{enumerate}
  \tightlist
  \item
    the \passthrough{\lstinline!ndarray.resize!} method modifies the
    array itself
  \end{enumerate}
\end{enumerate}

\begin{lstlisting}[language=Python]
import numpy as np
a = np.floor(10 * np.random.random((3,4)))
a.shape
\end{lstlisting}

\begin{lstlisting}[language=Python]
a.ravel()
\end{lstlisting}

\begin{lstlisting}[language=Python]
a.T
\end{lstlisting}

\begin{lstlisting}[language=Python]
a.reshape(2,6)
\end{lstlisting}

\begin{lstlisting}[language=Python]
a.resize(2,6)

"""END"""
\end{lstlisting}

\end{frame}

\begin{frame}[fragile]{Stacking together different arrays}
\protect\hypertarget{stacking-together-different-arrays}{}

\begin{enumerate}
\tightlist
\item
  \passthrough{\lstinline!hstack, vstack!}
\item
  \passthrough{\lstinline!column\_stack, row\_stack!}
\item
  \passthrough{\lstinline!concatenate!}
\item
  \passthrough{\lstinline!c\_, r\_!}
\end{enumerate}

\end{frame}

\begin{frame}[fragile]{Splitting one array into several smaller ones}
\protect\hypertarget{splitting-one-array-into-several-smaller-ones}{}

\begin{enumerate}
\tightlist
\item
  \passthrough{\lstinline!hsplit!}
\item
  \passthrough{\lstinline!vsplit!}
\item
  \passthrough{\lstinline!array\_split!}
\end{enumerate}

\end{frame}

\hypertarget{fancy-indexing-and-index-tricks}{%
\section{Fancy indexing and index
tricks}\label{fancy-indexing-and-index-tricks}}

\begin{frame}[fragile]{Indexing with Arrays of Indices-1D}
\protect\hypertarget{indexing-with-arrays-of-indices-1d}{}

\begin{enumerate}
\tightlist
\item
  使用 \passthrough{\lstinline!np.array!}
  对象作为索引下标可以取非连续元素
\end{enumerate}

\begin{lstlisting}[language=Python]
a = np.arange(12) ** 2 # the first 12 square numbers
i = np.array( [ 1,1,3,8,5 ] ) # an array of indices
a[i] # the elements of a at the positions i
\end{lstlisting}

\begin{lstlisting}[language=Python]
np.array([ 1, 1, 9, 64, 25])
\end{lstlisting}

\begin{lstlisting}[language=Python]
j = np.array( [ [ 3, 4], [ 9, 7 ] ] )
a[j]
\end{lstlisting}

\begin{lstlisting}[language=Python]
"""END"""
\end{lstlisting}

\end{frame}

\begin{frame}[fragile]{Indexing with Arrays of Indices-2D}
\protect\hypertarget{indexing-with-arrays-of-indices-2d}{}

\begin{enumerate}
\tightlist
\item
  We can also give indexes for more than one dimension. The arrays of
  indices for each dimension must have the same shape.
\item
  Naturally, we can put i and j in a sequence (say a list) and then do
  the indexing with the list.
\end{enumerate}

\begin{lstlisting}[language=Python]
a = np.arange(12).reshape(3,4)
i = np.array([[0,1], [1,2]])
j = np.array([[2,1], [3,3]])

a[i]
\end{lstlisting}

\begin{lstlisting}[language=Python]
a[i,j]
\end{lstlisting}

\begin{lstlisting}[language=Python]
a[i, 2]
\end{lstlisting}

\begin{lstlisting}[language=Python]
a[:,j]
\end{lstlisting}

\begin{lstlisting}[language=Python]
L = [i,j]
a[L]
\end{lstlisting}

\begin{lstlisting}[language=Python]
"""END"""
\end{lstlisting}

\end{frame}

\begin{frame}[fragile]{Indexing with Boolean Arrays}
\protect\hypertarget{indexing-with-boolean-arrays}{}

\begin{enumerate}
\tightlist
\item
  use boolean arrays that have the same shape as the original array
\item
  for each dimension of the array we give a 1D boolean array selecting
  the slices we want
\item
  Note that the length of the 1D boolean array must coincide with the
  length of the dimension (or axis) you want to slice.
\end{enumerate}

\begin{lstlisting}[language=Python]
a = np.arange(12).reshape(3,4)
b=a>4
a[b]
\end{lstlisting}

\begin{lstlisting}[language=Python]
a[b]=0

a = np.arange(12).reshape(3,4)
b1 = np.array([False,True,True])
b2 = np.array([True,False,True,False])
a[b1,:]
\end{lstlisting}

\begin{lstlisting}[language=Python]
a[b1]
\end{lstlisting}

\begin{lstlisting}[language=Python]
a[:,b2]
\end{lstlisting}

\begin{lstlisting}[language=Python]
a[b1,b2]
\end{lstlisting}

\begin{lstlisting}[language=Python]
"""END"""
\end{lstlisting}

\end{frame}

\begin{frame}[fragile]{Indexing with strings}
\protect\hypertarget{indexing-with-strings}{}

\begin{enumerate}
\tightlist
\item
  Structured arrays are ndarrays whose datatype is a composition of
  simpler datatypes organized as a sequence of named fields.
\item
  You can access and modify individual fields of a structured array by
  indexing with the field name.
\item
  One can index and assign to a structured array with a multi-field
  index, where the index is a list of field names.
\end{enumerate}

\begin{lstlisting}[language=Python]
x = np.array([('Rex', 9, 81.0), ('Fido', 3, 27.0)],
             dtype=[('name', 'U10'), ('age', 'i4'), ('weight', 'f4')])
x['name']
\end{lstlisting}

\begin{lstlisting}[language=Python]
x[['name', 'age']]
\end{lstlisting}

\begin{lstlisting}[language=Python]
"""END"""
\end{lstlisting}

\end{frame}

\begin{frame}[fragile]{The ix\textsubscript{()} function}
\protect\hypertarget{the-ix-function}{}

\begin{enumerate}
\tightlist
\item
  The ix\_ function can be used to combine different vectors so as to
  obtain the result for each n-uplet.(类似于R中的
  \passthrough{\lstinline!expand.grid!} 函数)
\item
  For example, if you want to compute all the a+b*c for all the triplets
  taken from each of the vectors a, b and c:
\end{enumerate}

\begin{lstlisting}[language=Python]
a = np.array([2,3,4,5])
b = np.array([8,5,4])
c = np.array([5,4,6,8,3])
ax,bx,cx = np.ix_(a,b,c)

result = ax+bx * cx
result
\end{lstlisting}

\begin{lstlisting}[language=Python]
"""END"""
\end{lstlisting}

\end{frame}

\hypertarget{universal-functions-ufunc}{%
\section{通用函数UNIVERSAL FUNCTIONS
(UFUNC)}\label{universal-functions-ufunc}}

\begin{frame}{简介}
\protect\hypertarget{section-1}{}

\begin{enumerate}
\tightlist
\item
  A universal function (or ufunc for short) is a function that operates
  on ndarrays in an element-by-element fashion.
\item
  That is, a ufunc is a ``vectorized'' wrapper for a function that takes
  a fixed number of specific inputs and produces a fixed number of
  specific outputs.
\item
  In NumPy, universal functions are instances of the numpy.ufunc class.
\item
  Many of the built-in functions are implemented in compiled C code.
\end{enumerate}

\end{frame}

\begin{frame}{常见通用函数}
\protect\hypertarget{section-2}{}

\begin{enumerate}
\tightlist
\item
  Math operations
\item
  Trigonometric functions
\item
  Floating functions
\item
  。。。
\end{enumerate}

\end{frame}

\hypertarget{section-3}{%
\section{统计功能}\label{section-3}}

\begin{frame}{Order statistics}
\protect\hypertarget{order-statistics}{}

\begin{longtable}[]{@{}ll@{}}
\toprule
函数 & 功能\tabularnewline
\midrule
\endhead
amin(a\(, axis, out, keepdims\)) & Return the minimum of an array or
minimum along an axis.\tabularnewline
amax(a\(, axis, out, keepdims\)) & Return the maximum of an array or
maximum along an axis\tabularnewline
nanmin(a\(, axis, out, keepdims\)) & Return minimum of an array or
minimum along an axis, ignoring any NaNs.\tabularnewline
nanmax(a\(, axis, out, keepdims\)) & Return the maximum of an array or
maximum along an axis, ignoring any NaNs.\tabularnewline
ptp(a\(, axis, out\)) & Range of values (maximum - minimum) along an
axis.\tabularnewline
percentile(a, q\(, axis, out, ...\)) & Compute the qth percentile of the
data along the specified axis\tabularnewline
nanpercentile(a, q\(, axis, out, ...\)) & Compute the qth percentile of
the data along the specified axis, while ignoring nan
values.\tabularnewline
\bottomrule
\end{longtable}

\end{frame}

\begin{frame}{Averages and variances}
\protect\hypertarget{averages-and-variances}{}

\begin{longtable}[]{@{}ll@{}}
\toprule
函数 & 功能\tabularnewline
\midrule
\endhead
median(a\(, axis, out, overwrite~input~, keepdims\)) & Compute the
median along the specified axis.\tabularnewline
average(a\(, axis, weights, returned\)) & Compute the weighted average
along the specified axis.\tabularnewline
mean(a\(, axis, dtype, out, keepdims\)) & Compute the arithmetic mean
along the specified axis.\tabularnewline
std(a\(, axis, dtype, out, ddof, keepdims\)) & Compute the standard
deviation along the specified axis.\tabularnewline
var(a\(, axis, dtype, out, ddof, keepdims\)) & Compute the variance
along the specified axis.\tabularnewline
nanmedian(a\(, axis, out, overwrite~input~, ...\)) & Compute the median
along the specified axis, while ignoring NaNs.\tabularnewline
nanmean(a\(, axis, dtype, out, keepdims\)) & Compute the arithmetic mean
along the specified axis, ignoring NaNs.\tabularnewline
nanstd(a\(, axis, dtype, out, ddof, keepdims\)) & Compute the standard
deviation along the specified axis,while ignoring NaNs.\tabularnewline
nanvar(a\(, axis, dtype, out, ddof, keepdims\)) & Compute the variance
along the specified axis, while ignoring NaNs.\tabularnewline
\bottomrule
\end{longtable}

注:几何平均数,调和平均数函数在scipy中

\end{frame}

\begin{frame}{Correlating}
\protect\hypertarget{correlating}{}

\begin{longtable}[]{@{}ll@{}}
\toprule
函数 & 功能\tabularnewline
\midrule
\endhead
corrcoef(x\(, y, rowvar, bias, ddof\)) & Return Pearson product-moment
correlation coefficients.\tabularnewline
correlate(a, v\(, mode\)) & Cross-correlation of two 1-dimensional
sequences.\tabularnewline
cov(m\(, y, rowvar, bias, ddof, fweights, ...\)) & Estimate a covariance
matrix, given data and weights.\tabularnewline
\bottomrule
\end{longtable}

\end{frame}

\begin{frame}{Histograms}
\protect\hypertarget{histograms}{}

\end{frame}

\hypertarget{linear-algebra-numpy.linalg}{%
\section{Linear algebra
(numpy.linalg)}\label{linear-algebra-numpy.linalg}}

\begin{frame}{常见矩阵运算- Matrix and vector products}
\protect\hypertarget{matrix-and-vector-products}{}

\scriptsize

\begin{longtable}[]{@{}ll@{}}
\toprule
函数 & 功能\tabularnewline
\midrule
\endhead
dot(a, b) & Dot product of two arrays.\tabularnewline
vdot(a, b) & Return the dot product of two vectors.\tabularnewline
inner(a, b) & Inner product of two arrays.\tabularnewline
outer(a, b) & Compute the outer product of two vectors.\tabularnewline
linalg.matrix\textsubscript{power}(M, n) & Raise a square matrix to the
(integer) power n.\tabularnewline
kron(a, b) & Kronecker product of two arrays.\tabularnewline
\bottomrule
\end{longtable}

\end{frame}

\begin{frame}{常见矩阵运算- Decompositions}
\protect\hypertarget{decompositions}{}

\scriptsize

\begin{longtable}[]{@{}ll@{}}
\toprule
函数 & 功能\tabularnewline
\midrule
\endhead
linalg.cholesky(a) & Cholesky decomposition.\tabularnewline
linalg.qr(a) & Compute the qr factorization of a matrix.\tabularnewline
linalg.svd(a) & Singular Value Decomposition.\tabularnewline
linalg.eig(a) & Compute the eigenvalues and right eigenvectors of a
square array.\tabularnewline
scipy.linalg.lu(a) & Compute pivoted LU decompostion of a
matrix.\tabularnewline
\bottomrule
\end{longtable}

\end{frame}

\begin{frame}{常见矩阵运算- Norms and other numbers}
\protect\hypertarget{norms-and-other-numbers}{}

\scriptsize

\begin{longtable}[]{@{}ll@{}}
\toprule
函数 & 功能\tabularnewline
\midrule
\endhead
linalg.norm(x) & Matrix or vector norm.\tabularnewline
linalg.cond(x) & Compute the condition number of a
matrix.\tabularnewline
linalg.det(a) & Compute the determinant of an array.\tabularnewline
linalg.matrix\textsubscript{rank}(M) & Return matrix rank of array using
SVD method\tabularnewline
trace(a) & Return the sum along diagonals of the array.\tabularnewline
\bottomrule
\end{longtable}

\end{frame}

\begin{frame}{常见矩阵运算- Solving equations and inverting matrices}
\protect\hypertarget{solving-equations-and-inverting-matrices}{}

\scriptsize

\begin{longtable}[]{@{}ll@{}}
\toprule
函数 & 功能\tabularnewline
\midrule
\endhead
linalg.solve(a, b) & Solve a linear matrix equation, or system of linear
scalar equations.\tabularnewline
linalg.tensorsolve(a, b) & Solve the tensor equation a x = b for
x.\tabularnewline
linalg.lstsq(a, b) & Return the least-squares solution to a linear
matrix equation.\tabularnewline
linalg.inv(a) & Compute the (multiplicative) inverse of a
matrix.\tabularnewline
linalg.pinv(a) & Compute the (Moore-Penrose) pseudo-inverse of a
matrix.\tabularnewline
\bottomrule
\end{longtable}

\end{frame}

\hypertarget{random-sampling-numpy.random}{%
\section{Random sampling
(numpy.random)}\label{random-sampling-numpy.random}}

\begin{frame}[fragile]{简介}
\protect\hypertarget{section-4}{}

\begin{enumerate}
\tightlist
\item
  下面这些函数主要用于随机抽样和生成随机数字,关于概率,分位点等计算见
  \passthrough{\lstinline!scipy.stats!} 模块
\item
  下面函数都以 \passthrough{\lstinline!np.random.!} 开始
\item
  有很多功能相同名字不同的函数
\end{enumerate}

\end{frame}

\begin{frame}{Simple random data}
\protect\hypertarget{simple-random-data}{}

\scriptsize

\begin{longtable}[]{@{}ll@{}}
\toprule
函数 & 功能\tabularnewline
\midrule
\endhead
rand(d0, d1, \ldots, dn) & Random values in a given
shape.\tabularnewline
randn(d0, d1, \ldots, dn) & Return a sample (or samples) from the
``standard normal'' distribution.\tabularnewline
randint(low\(, high, size, dtype\)) & Return random integers from low
(inclusive) to high (exclusive).\tabularnewline
random\textsubscript{integers}(low\(, high, size\)) & Random integers of
type np.int between low and high, inclusive.\tabularnewline
random\textsubscript{sample}(\(size\)) & Return random floats in the
half-open interval {[}0.0, 1.0).\tabularnewline
random(\(size\)) & Return random floats in the half-open interval
{[}0.0, 1.0).\tabularnewline
ranf(\(size\)) & Return random floats in the half-open interval {[}0.0,
1.0).\tabularnewline
sample(\(size\)) & Return random floats in the half-open interval
{[}0.0, 1.0).\tabularnewline
choice(a\(, size, replace, p\)) & Generates a random sample from a given
1-D array\tabularnewline
bytes(length) & Return random bytes.\tabularnewline
\bottomrule
\end{longtable}

\end{frame}

\begin{frame}{Permutations}
\protect\hypertarget{permutations}{}

\scriptsize

\begin{longtable}[]{@{}ll@{}}
\toprule
函数 & 功能\tabularnewline
\midrule
\endhead
shuffle(x) & Modify a sequence in-place by shuffling its
contents.\tabularnewline
permutation(x) & Randomly permute a sequence, or return a permuted
range.\tabularnewline
\bottomrule
\end{longtable}

\end{frame}

\begin{frame}{Distribution}
\protect\hypertarget{distribution}{}

\scriptsize

\begin{longtable}[]{@{}ll@{}}
\toprule
函数 & 功能\tabularnewline
\midrule
\endhead
beta(a, b\(, size\)) & Draw samples from a Beta
distribution.\tabularnewline
binomial(n, p\(, size\)) & Draw samples from a binomial
distribution.\tabularnewline
chisquare(df\(, size\)) & Draw samples from a chi-square
distribution.\tabularnewline
dirichlet(alpha\(, size\)) & Draw samples from the Dirichlet
distribution.\tabularnewline
exponential(\(scale, size\)) & Draw samples from an exponential
distribution.\tabularnewline
f(dfnum, dfden\(, size\)) & Draw samples from an F
distribution.\tabularnewline
gamma(shape\(, scale, size\)) & Draw samples from a Gamma
distribution.\tabularnewline
geometric(p\(, size\)) & Draw samples from the geometric
distribution.\tabularnewline
gumbel(\(loc, scale, size\)) & Draw samples from a Gumbel
distribution.\tabularnewline
hypergeometric(ngood, nbad, nsample\(, size\)) & Draw samples from a
Hypergeometric distribution.\tabularnewline
laplace(\(loc, scale, size\)) & Draw samples from the Laplace or double
exponential distribution with specified location (or mean) and scale
(decay).\tabularnewline
logistic(\(loc, scale, size\)) & Draw samples from a logistic
distribution.\tabularnewline
lognormal(\(mean, sigma, size\)) & Draw samples from a log-normal
distribution.\tabularnewline
logseries(p\(, size\)) & Draw samples from a logarithmic series
distribution.\tabularnewline
\bottomrule
\end{longtable}

\end{frame}

\begin{frame}{Distributions}
\protect\hypertarget{distributions}{}

\scriptsize

\begin{longtable}[]{@{}ll@{}}
\toprule
函数 & 功能\tabularnewline
\midrule
\endhead
multivariate\textsubscript{normal}(mean, cov\(, size, ...\)) & Draw
random samples from a multivariate normal distribution.\tabularnewline
negative\textsubscript{binomial}(n, p\(, size\)) & Draw samples from a
negative binomial distribution.\tabularnewline
noncentral\textsubscript{chisquare}(df, nonc\(, size\)) & Draw samples
from a noncentral chi-square distribution.\tabularnewline
noncentral\textsubscript{f}(dfnum, dfden, nonc\(, size\)) & Draw samples
from the noncentral F distribution.\tabularnewline
normal(\(loc, scale, size\)) & Draw random samples from a normal
(Gaussian) distribution.\tabularnewline
pareto(a\(, size\)) & Draw samples from a Pareto II or Lomax
distribution with specified shape.\tabularnewline
poisson(\(lam, size\)) & Draw samples from a Poisson
distribution.\tabularnewline
power(a\(, size\)) & Draws samples in \(0, 1\) from a power distribution
with positive exponent a - 1.\tabularnewline
rayleigh(\(scale, size\)) & Draw samples from a Rayleigh
distribution.\tabularnewline
standard\textsubscript{cauchy}(\(size\)) & Draw samples from a standard
Cauchy distribution with mode = 0.\tabularnewline
standard\textsubscript{exponential}(\(size\)) & Draw samples from the
standard exponential distribution.\tabularnewline
standard\textsubscript{gamma}(shape\(, size\)) & Draw samples from a
standard Gamma distribution.\tabularnewline
standard\textsubscript{normal}(\(size\)) & Draw samples from a standard
Normal distribution (mean=0, stdev=1).\tabularnewline
standard\textsubscript{t}(df\(, size\)) & Draw samples from a standard
Student's t distribution with df degrees of freedom.\tabularnewline
triangular(left, mode, right\(, size\)) & Draw samples from the
triangular distribution over the interval
\(left, right\).\tabularnewline
uniform(\(low, high, size\)) & Draw samples from a uniform
distribution.\tabularnewline
vonmises(mu, kappa\(, size\)) & Draw samples from a von Mises
distribution.\tabularnewline
wald(mean, scale\(, size\)) & Draw samples from a Wald, or inverse
Gaussian, distribution.\tabularnewline
weibull(a\(, size\)) & Draw samples from a Weibull
distribution.\tabularnewline
zipf(a\(, size\)) & Draw samples from a Zipf
distribution.\tabularnewline
\bottomrule
\end{longtable}

\end{frame}

\begin{frame}{Random generator}
\protect\hypertarget{random-generator}{}

\scriptsize

\begin{longtable}[]{@{}ll@{}}
\toprule
函数 & 功能\tabularnewline
\midrule
\endhead
RandomState(\(seed\)) & Container for the Mersenne Twister pseudo-random
number generator.\tabularnewline
seed(\(seed\)) & Seed the generator.\tabularnewline
get\textsubscript{state}() & Return a tuple representing the internal
state of the generator.\tabularnewline
set\textsubscript{state}(state) & Set the internal state of the
generator from a tuple.\tabularnewline
\bottomrule
\end{longtable}

\end{frame}

\hypertarget{functions-and-methods-overview}{%
\section{Functions and Methods
Overview}\label{functions-and-methods-overview}}

\begin{frame}{Functions and Methods Overview}
\protect\hypertarget{functions-and-methods-overview-1}{}

\begin{enumerate}
\item
  \textbf{Array Creation}:

  arange, array, copy, empty, empty\textsubscript{like}, eye, fromfile,
  fromfunction,identity, linspace, logspace, mgrid, ogrid, ones,
  ones\textsubscript{like}, r, zeros, zeros\textsubscript{like}
\item
  \textbf{Conversions}:

  ndarray.astype, atleast\textsubscript{1d}, atleast\textsubscript{2d},
  atleast\textsubscript{3d}, mat
\item
  \textbf{Manipulations}:

  array\textsubscript{split}, column\textsubscript{stack}, concatenate,
  diagonal, dsplit, dstack, hsplit,hstack, ndarray.item, newaxis, ravel,
  repeat, reshape, resize, squeeze, swapaxes, take, transpose, vsplit,
  vstack
\end{enumerate}

\end{frame}

\begin{frame}{Functions and Methods Overview}
\protect\hypertarget{functions-and-methods-overview-2}{}

\begin{enumerate}
\item
  \textbf{Questions}:

  all, any, nonzero, where
\item
  \textbf{Ordering}:

  argmax, argmin, argsort, max, min, ptp, searchsorted, sort
\item
  \textbf{Operations}:

  choose, compress, cumprod, cumsum, inner, ndarray.fill, imag, prod,
  put, putmask,real, sum
\item
  \textbf{Basic Statistics}:

  cov, mean, std, var
\item
  \textbf{Basic Linear Algebra}:

  cross, dot, outer, linalg.svd, vdot
\end{enumerate}

\end{frame}

\hypertarget{other-subpackages}{%
\section{Other Subpackages}\label{other-subpackages}}

\begin{frame}{Numpy中的其他常用模块}
\protect\hypertarget{numpy-1}{}

\begin{enumerate}
\tightlist
\item
  String operations
\item
  Datetime Support Functions
\item
  Discrete Fourier Transform (numpy.fft)
\item
  Financial functions
\item
  Functional programming
\item
  Logic functions
\item
  Mathematical functions
\item
  Matrix library (numpy.matlib)
\item
  numpy.polynomial package
\item
  具体用法和更多模块可以参考 Numpy reference\[numpy-ref.pdf\]中的
  Routines 内容
\end{enumerate}

\end{frame}




\end{document}


